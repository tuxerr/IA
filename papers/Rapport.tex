\documentclass[12pt]{article}
\usepackage{amsmath}
%\usepackage{graphics}
\usepackage[french]{babel}
\usepackage[utf8x]{inputenc}
\usepackage{vmargin} 
\usepackage{graphicx}  
\usepackage{amsmath}
\setpapersize{A4}
\setmarginsrb   %default values
{25mm}  % leftmargin : 30mm
{20mm}  % topmargin
{25mm}  % rightmargin
{20mm}  % bottommargin : 25mm
{12pt}  % headheight
{5mm}   % headsep
{12pt}  % footheight
{10mm}  % footskip

\date{7 janvier 2013}

\title{Projet d'intelligence artificielle :\\Simulation de Société d'Agents Coopératifs}

\author{Jason \textsc{Crombez} \\ Marine \textsc{Lavaux} \\ Rémi \textsc{Palandri}}
	
\begin{document}
	
\maketitle

\newpage
	
\section{Sujet}
	
Ce projet consiste à définir une société d'agents où chaque agent aura un rôle.
Pour les agents humains ce rôle pourra être :\\

	\begin{itemize}

	\item cultivateur
    	\item éleveur
    	\item chasseur
    	\item porteurs d'eau
    	\item constructeur
    	\item cuisinier
    	\item chef de clan (un seul agent)\\

	\end{itemize}

Il y aura également des agents animaux qui pourront être soit sauvages soit 
domesticables.\\

Ce clan et ces animaux vivront dans un univers clos, représentable par une île 
(grille 2D rectangle), constitué d'une plaine dont une partie sera occupée par 
le clan (village) et d'une jungle.\\

Cet univers sera soumis à des intempéries (dont on n'est pas obligé de 
représenter la nature exacte : foudre, éboulements, innondations, ...) et à des 
antagonismes (chasses d'animaux qui veulent fuir les humains et survivre à 
leurs attaques, attaques par les animaux des humains qui veulent fuir et 
survivre eux aussi, ...) dont l'effet ultime sera toujours la diminution (la 
mort), dans une certaine proportion (simulée par des tirages aléatoires), du 
nombre d'agents (humains ou animaux). Les agents auront néanmoins la faculté 
de se reproduire si certaines conditions sont satisfaites (conditions de 
ressources alimentaires, ...).\\

Les agents animaux chercheront essentiellement à se nourrir et à se reproduire. 
Leur capacité à se nourrir pourra être entravée par le manque d'étendue dont 
ils peuvent profiter (surface de l'ile moins celle du village et des cultures). 
Leur capacité à se reproduire pourra dépendre de leur capacité à se nourrir et 
de leur nombre (plus ils sont nombreux plus ils peuvent se rencontrer).\\

Les objets abris peuvent servir à faire diminuer les probabilités de mort des 
agents humains, les huttes de stockage servent de réserves et permettent de 
survivre même si le temps ne permet ni les récoltes, ni la chasse. Ces objets 
peuvent être détériorés et devront être reconstruits ou réparés.\\

Toutes les activités prennent du temps et de l'énergie vitale à tous les agents 
(humains ou animaux). Ils doivent tous se nourrir et se reposer. Ces 
caractéristiques peuvent aussi être simulées par différentes jauges.\\

L'agent chef de clan est un peu particulier dans la mesure où il pourrait 
décider de certaines affectations comme décider combien d'agents de tel type 
il faut consacrer à la tâche qui leur incombe, décider quels tâches accomplir, 
combien de temps y consacrer, etc (voir quels sont les paramètres utiles et 
sur lesquels le chef de clan pourra jouer).\\

L'unité de surface cultivable rapporte une certaine quantité de nourriture 
végétale pour autant qu'elle soit cultivée pendant un certain temps par un 
certain nombre d'agents cultivateurs. Il en est de même concernant l'élevage 
des animaux domestiques.\\

% dans les consignes sur le site d'evrard il conseille \section{Objectifs}
% mais je ne vois pas quoi mettre dedans

\section{Analyse du sujet}

	\subsection{L'univers}

L'univers est clos, il s'agit d'une île. Celle-ci est est générée 
automatiquement à l'aide de Perlin Noise, c'est également le cas en ce qui 
concerne les différents types de terrains (forêts, lacs, ...) les 
coefficients peuvent être ajustés afin de créer des cartes ayant leur 
spécificités propres.

% detail perlin noise ?

La carte est divisée en cellules unitaires, celles-ci contiennent les 
informations relatives au terrain, aux agents et bâtiments présents sur la 
cellule.
					
\end{document}
	
	
